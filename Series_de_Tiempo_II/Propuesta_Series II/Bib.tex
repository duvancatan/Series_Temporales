\begin{thebibliography}{}

\addcontentsline{toc}{chapter}{Bibliograf\'ia}

\bibitem{BAI} Bai, J and Ng, S. (2002) Determining the number of factors in approximate factor models. \emph{Econometrica} 70(1), 191-221.

\bibitem{CH} Chamberlain, G. and Rothschild, M. (1983). Arbitrage, factor structure, and mean-variance analysis on large asset markets. \emph{Econometrica} 51 (5), 1281�1304.

\bibitem{CP} Correal M., Pe�a, D. (2008). Modelo factorial din�mico threshold. \emph{Revista Colombiana de Estad�stica}. Diciembre 2008, volumen 31, no. 2, pp. 183 a 192.

\bibitem{D} Dahlhaus R. (2000). A likelihood approximation for locally stationary processes. \emph{The Annals of Statistics}, 28:1762-94.

\bibitem{D1} Dahlhaus R., Eichler M., Sandkuhler J. (1999). Identification of synaptic connections in neural ensembles by graphical models. \emph{J Neurosci Methods}, 77(1):93-107.

\bibitem{D3} Dahlhaus, R. (1997). Fitting time series models to nonstationary processes. \emph{The Annals of Statistics} 25, 1�37.

\bibitem{EMS} Eichler, M., Motta, G., and von Sachs, R. (2011). Fitting dynamic factor models to nonstationary time series. \emph{Journal of Econometrics} 163, 51-70.

\bibitem{F} Forni, M., Hallin, M., Lippi, M., Reichlin, L., (2000). The generalized dynamic factor model: identification and estimation. \emph{Rev. Econ. Statist.} 82, 540�554.

\bibitem{LY} Lam, C. and Yao, Q. (2012). Factor modeling for high-dimensional time series: inference for the number of factors. \emph{Annals of Statistics} 40, 694-726.

\bibitem{MOT} Motta, G. Hafner, C.M. and von Sachs, R (2008). Locally stationary factor models: identification and nonparametric estimation. \emph{Econometric Theory} 27, 2011, 1279--1319.

\bibitem{N} Nieto, F., Pe�a, D., Saboy�, D., (2015). Common Seasonality in Multivariate Time Series. Universidad Nacional de Colombia.

\bibitem{N} Ombao, H ., and Ho, M., (2006). Time-dependent frequency domain principal components analysis of multichannel non-stationary signals. \emph{Computational and Data Analysis} 50 (2006) 2339-2360. 

\bibitem{PB} Pe�a, D. Box, G. (1987). Identifying a simplifying structure in time series. \emph{Journal of the American Statistical Association} 82, 836�843.

\bibitem{PP} Pe�a, D. Poncela, P. (2004). Forecasting with Nonstationary Dynamic Factor Models. \emph{Journal of Econometrics} 119, 291�321.

\bibitem{PP2006} Pe�a, D., Poncela, P., (2006). Nonstationary dynamic factor analysis. \emph{Journal of Statistical Planning and Inference} 136 (2006) 1237 - 1257

\bibitem{RP} Rodr�guez-Poo, J. M. and Linton, O. (2001). Nonparametric factor analysis of residual time series. \emph{Test} 10, 161�182.

\bibitem{S} Sato, J., (2007). Modelo Autoregressivo Vetorial com Coeficientes Variantes no Tempo e Aplica��es em RMf. Tese Doutorado. IME-USP. 

\bibitem{SW} Stock, J.H., Watson, M.W. (1988). Testing for common trends. \emph{Journal of the American Statistical Association} 83, 1097�1107.

\bibitem{TT} Tiao, G.C., Tsay, R.S. (1989). Model specication in multivariate time series. \emph{Journal of the Royal Statistical Society} B 51, 157�213.

\bibitem{T} Tsay, Ruey (2014). Multivariate Time Series Analysis With R and Financial Applications. 









%\bibitem{G} Geweke (1977)



%\bibitem{JW} Johnson e Wichern (2007)





%\bibitem{LYB} Lam, Yao, e Bathia (2011) 



%\bibitem{F} Forni et al. (2000, 2004, 2005) 





%\bibitem{sah} An Chen, Antoon Pelsser, Michel Vellekoop (2011): ``Modeling non-monotone risk aversion using SAHARA utility functions'', \emph{Journal of Economic Theory} 146 (2011) 2075-2092.



\end{thebibliography}