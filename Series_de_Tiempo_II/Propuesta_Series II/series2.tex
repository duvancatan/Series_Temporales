\documentclass[12pt,twoside,a4paper]{book}
\usepackage{ucs}
\usepackage[latin1,applemac]{inputenc} %Mac
%\usepackage[utf8x]{inputenc} % Linux
%\usepackage[latin1]{inputenc} %Windows
\usepackage[T1]{fontenc}
\usepackage[portuges,brazil,brazilian]{babel}
\usepackage[pdftex]{graphicx}
\usepackage{setspace}
\usepackage{indentfirst}
\usepackage{makeidx}
\usepackage[nottoc]{tocbibind}
\usepackage{courier}
\usepackage{type1cm}
\usepackage{listings}
\usepackage{titletoc}
\usepackage{afterpage,mathrsfs}
\usepackage{latexsym,amssymb,amsmath,bbm,amsxtra,amstext}
\usepackage{palatino}
\usepackage{pstricks,pst-node}
\usepackage{amsfonts,epsfig,multicol,anysize}
\usepackage{multicol}
\usepackage{float}
\usepackage{longtable}
\usepackage{calrsfs}
\usepackage{multirow}
\usepackage[fixlanguage]{babelbib}
\usepackage[font=small,format=plain,labelfont=bf,up,textfont=it,up]{caption}
%\usepackage[usenames,svgnames,dvipsnames]{xcolor}
\usepackage[a4paper,top=3.54cm,bottom=3.0cm,left=2.54cm,right=2.54cm]{geometry} 
\usepackage{cite}
\usepackage[pdftex,plainpages=false,pdfpagelabels,pagebackref,colorlinks=true,citecolor=black,linkcolor=black,urlcolor=black,filecolor=black,bookmarksopen=true]{hyperref} 



%\usepackage[footnotesize]{caption}
%\usepackage{ragged2e}
%\usepackage{multirow}
%\usepackage[pdftex]{graphicx}
%\usepackage[left=2.5cm,top=2cm,right=2.5cm,bottom=2.5cm]{geometry}
%\usepackage[latin1]{inputenc}
%\usepackage{amsmath} %Para referencia eqref()
%\usepackage[portuguese]{babel}
%\usepackage{geometry} % see geometry.pdf on how to lay out the page. There's lots.
%\geometry{a4paper} % or letter or a5paper or ... etc
% \geometry{landscape} % rotated page geometry




\thispagestyle{empty} 

\usepackage[all]{hypcap}
%\bibpunct{(}{)}{;}{a}{\hspace{-0.7ex},}{,} % estilo de cita��o. Veja 
\usepackage[square,comma,numbers]{natbib}
\bibliographystyle{IEEEtran}


\fontsize{60}{62}\usefont{OT1}{cmr}{m}{n}{\selectfont}





\usepackage{amsmath} 
\usepackage{soul} 
\usepackage{color} 
\usepackage{amssymb} 

\def\tr{\mathop{\rm tr}\nolimits}
\def\diag{\mathop{\rm diag}\nolimits}
\def\tr{\mathop{\rm tr}\nolimits}
\def\rank{\mathop{\rm rank}\nolimits}
\def\vec{\mathop{\rm vec}\nolimits}
\def\vecp{\mathop{\rm vecp}\nolimits}
\def\vol{\mathop{\rm Vol}\nolimits}
\def\etr{\mathop{\rm etr}\nolimits}
%\def\Rel{\mathop{\rm Re}\nolimits}
\def\cov{\mathop{\rm cov}\nolimits}
\def\corr{\mathop{\rm corr}\nolimits}
\def\func #1{\mathop{\rm #1}\nolimits}%
\def\blackbox{\ \rule{0.5em}{0.5em}}


\def\r{\mathds{R}}
\def\R{\mathbb{R}}
\DeclareMathAlphabet{\mathpzc}{OT1}{pzc}{m}{it}

% ---------------------------------------------------------------------------- %
% Cabe�alhos similares ao TAOCP de Donald E. Knuth
\usepackage{fancyhdr}
\pagestyle{fancy}
\fancyhf{}
\renewcommand{\chaptermark}[1]{\markboth{\MakeUppercase{#1}}{}}
\renewcommand{\sectionmark}[1]{\markright{\MakeUppercase{#1}}{}}
\renewcommand{\headrulewidth}{1pt}
\renewcommand{\footrulewidth}{1pt}
\renewcommand{\thesection}{\@arabic\c@section}
% ---------------------------------------------------------------------------- %

\renewcommand*\thesection{\arabic{section}}
\renewcommand*\thesubsection{\arabic{subsection}}
%\renewcommand{\qedsymbol}{$\blacksquare$}

\graphicspath{{./figuras/}}  
\frenchspacing               
\urlstyle{same}
\makeindex
\raggedbottom
\fontsize{60}{62}\usefont{OT1}{cmr}{m}{n}{\selectfont}
\cleardoublepage
\normalsize


\lstset{
language=Java,
basicstyle=\footnotesize,
numbers=left,
numberstyle=\footnotesize,
stepnumber=1,
numbersep=5pt,
showspaces=false,
showstringspaces=false,
showtabs=false,
frame=single,
framerule=0.6pt,
tabsize=2,
captionpos=b,
breaklines=true,
breakatwhitespace=false,
escapeinside={\%*}{*)},
backgroundcolor=\color[rgb]{1.0,1.0,1.0},
rulecolor=\color[rgb]{0.8,0.8,0.8},
extendedchars=true,
xleftmargin=10pt,
xrightmargin=10pt,
framexleftmargin=10pt,
framexrightmargin=10pt
}

\newtheorem{theorem}{Teorema}[section]
\newtheorem{proposition}{Proposi��o}[section]
\newtheorem{definition}{Defini��o}[section]
\newtheorem{corollary}{Corol�rio}[theorem]
\newtheorem{lemma}{Lema}[section]
\newtheorem{example}{Exemplo}[section]
\newtheorem{observation}{Observa��o}[section]
\newtheorem{algorithm}{Algoritmo}[section]
\newtheorem{remark}{Nota}[section]

\providecommand{\n}[1]{{\boldsymbol{#1}}}
\newcommand{\e}{\mbox{E}\ell} % elíptica
\newcommand{\mb}{\mbox{BCE}\ell} % Box-Cox elíptica
\newcommand{\mbn}{\mbox{BCN}} % Box-Cox normal
\newcommand{\mbt}{{\mbox{BC}}{\it{t}}} % Box-Cox t
\newcommand{\te}{\mbox{TE}\ell} % truncada elíptica
\newcommand{\ten}{\mbox{TN}} % truncada normal 
\newcommand{\tet}{{\mbox{T}}{\it t}} % truncada t
\newcommand{\esf}{\mbox{S}} % esférica
\newcommand{\bcs}{\mbox{BCS}} % Box-Cox simétrica
\newcommand{\ls}{\mbox{LS}} % log-simétrica
\newcommand{\tesf}{\mbox{TS}} % truncada esférica
\newcommand{\lel}{\mbox{LE}\ell} % log-elíptica



\begin{document}
\frontmatter 
% cabe�alho para as p�ginas das se��es anteriores ao cap�tulo 1 (frontmatter)
\fancyhead[LE]{{\footnotesize\rightmark}\hspace{2em}\thepage}
\setcounter{tocdepth}{2}
%\fancyhead[LE]{\thepage\hspace{2em}\footnotesize{\leftmark}}
%\fancyhead[RE,LO]{}
%\fancyhead[RO]{{\footnotesize\rightmark}\hspace{2em}\thepage}

%\onehalfspacing  % espa�amento

% ---------------------------------------------------------------------------- %
% CAPA
% Nota: O t�tulo para as disserta��es/teses do IME-USP devem caber em um 
% orif�cio de 10,7cm de largura x 6,0cm de altura que h� na capa fornecida pela SPG.
\thispagestyle{empty}

\begin{figure}[H]
\centering
\includegraphics[scale=1]{udea.png}
\end{figure}

\begin{center}
    \vspace*{1cm}
    \textbf{\LARGE{Universidad de Antioquia}}\\
    
    \vspace*{1cm}
    \textbf{\Large{Instituto de Matem�ticas}}
    
    \vskip 2cm
    \Large{\textsc{
    Propuesta del curso \\
    Series de Tiempo II\\
    como electiva del Pregrado en Estad�stica\\[-0.25cm]}}
    
    
     \vspace{1.5cm}
    \textbf{\Large{Duv�n Humberto Cata�o Salazar}}
    
    \Large{Docente ocasional}\\
    
    
    \vskip 2cm
    \normalsize{Medell�n, 2018-2}
\end{center}

%\tableofcontents

%\listoftables  

%\mainmatter

%\fancyhead[RE,LO]{\thesection}

%\singlespacing 
%\onehalfspacing 

%\input glosario
\input introducao
%\input objetivos
%\input elipticas
%t\input elipticas-truncadas
%\input box-cox-elipticas
%\input aplicacao
%\input discussao

\renewcommand{\sectionmark}[1]{\markboth{\MakeUppercase{\appendixname\ \thesection}} {\MakeUppercase{#1}} }
\fancyhead[RE,LO]{}
%\appendix
%\include{apendice}
%\begin{thebibliography}{}

\addcontentsline{toc}{chapter}{Bibliograf\'ia}

\bibitem{BAI} Bai, J and Ng, S. (2002) Determining the number of factors in approximate factor models. \emph{Econometrica} 70(1), 191-221.

\bibitem{CH} Chamberlain, G. and Rothschild, M. (1983). Arbitrage, factor structure, and mean-variance analysis on large asset markets. \emph{Econometrica} 51 (5), 1281�1304.

\bibitem{CP} Correal M., Pe�a, D. (2008). Modelo factorial din�mico threshold. \emph{Revista Colombiana de Estad�stica}. Diciembre 2008, volumen 31, no. 2, pp. 183 a 192.

\bibitem{D} Dahlhaus R. (2000). A likelihood approximation for locally stationary processes. \emph{The Annals of Statistics}, 28:1762-94.

\bibitem{D1} Dahlhaus R., Eichler M., Sandkuhler J. (1999). Identification of synaptic connections in neural ensembles by graphical models. \emph{J Neurosci Methods}, 77(1):93-107.

\bibitem{D3} Dahlhaus, R. (1997). Fitting time series models to nonstationary processes. \emph{The Annals of Statistics} 25, 1�37.

\bibitem{EMS} Eichler, M., Motta, G., and von Sachs, R. (2011). Fitting dynamic factor models to nonstationary time series. \emph{Journal of Econometrics} 163, 51-70.

\bibitem{F} Forni, M., Hallin, M., Lippi, M., Reichlin, L., (2000). The generalized dynamic factor model: identification and estimation. \emph{Rev. Econ. Statist.} 82, 540�554.

\bibitem{LY} Lam, C. and Yao, Q. (2012). Factor modeling for high-dimensional time series: inference for the number of factors. \emph{Annals of Statistics} 40, 694-726.

\bibitem{MOT} Motta, G. Hafner, C.M. and von Sachs, R (2008). Locally stationary factor models: identification and nonparametric estimation. \emph{Econometric Theory} 27, 2011, 1279--1319.

\bibitem{N} Nieto, F., Pe�a, D., Saboy�, D., (2015). Common Seasonality in Multivariate Time Series. Universidad Nacional de Colombia.

\bibitem{N} Ombao, H ., and Ho, M., (2006). Time-dependent frequency domain principal components analysis of multichannel non-stationary signals. \emph{Computational and Data Analysis} 50 (2006) 2339-2360. 

\bibitem{PB} Pe�a, D. Box, G. (1987). Identifying a simplifying structure in time series. \emph{Journal of the American Statistical Association} 82, 836�843.

\bibitem{PP} Pe�a, D. Poncela, P. (2004). Forecasting with Nonstationary Dynamic Factor Models. \emph{Journal of Econometrics} 119, 291�321.

\bibitem{PP2006} Pe�a, D., Poncela, P., (2006). Nonstationary dynamic factor analysis. \emph{Journal of Statistical Planning and Inference} 136 (2006) 1237 - 1257

\bibitem{RP} Rodr�guez-Poo, J. M. and Linton, O. (2001). Nonparametric factor analysis of residual time series. \emph{Test} 10, 161�182.

\bibitem{S} Sato, J., (2007). Modelo Autoregressivo Vetorial com Coeficientes Variantes no Tempo e Aplica��es em RMf. Tese Doutorado. IME-USP. 

\bibitem{SW} Stock, J.H., Watson, M.W. (1988). Testing for common trends. \emph{Journal of the American Statistical Association} 83, 1097�1107.

\bibitem{TT} Tiao, G.C., Tsay, R.S. (1989). Model specication in multivariate time series. \emph{Journal of the Royal Statistical Society} B 51, 157�213.

\bibitem{T} Tsay, Ruey (2014). Multivariate Time Series Analysis With R and Financial Applications. 









%\bibitem{G} Geweke (1977)



%\bibitem{JW} Johnson e Wichern (2007)





%\bibitem{LYB} Lam, Yao, e Bathia (2011) 



%\bibitem{F} Forni et al. (2000, 2004, 2005) 





%\bibitem{sah} An Chen, Antoon Pelsser, Michel Vellekoop (2011): ``Modeling non-monotone risk aversion using SAHARA utility functions'', \emph{Journal of Economic Theory} 146 (2011) 2075-2092.



\end{thebibliography}

\backmatter \singlespacing   
%\bibliography{referencias} 


\end{document}