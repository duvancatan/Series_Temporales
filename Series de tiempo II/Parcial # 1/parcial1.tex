\documentclass[legalpaper,11pt]{article}

\pagestyle{empty}

\usepackage[ansinew]{inputenc}
\usepackage[spanish]{babel}
\usepackage[mathcal]{euscript}
\usepackage{amsmath,amsfonts,amssymb,theorem,latexsym,mathrsfs, %hyperref,
            epsfig, multicol,anysize,graphicx,enumitem,mdwlist}
\usepackage{graphicx}  
\usepackage{ragged2e}  
\usepackage{float}        

%%%%%%%%%%%%%%%%%%%%%%%%%%%%%%%%%

\marginsize{2cm}{2cm}{2cm}{2cm}

%%%%%%%%%%%%%%%%%%%%%%%%%%%%%%%%%%%%%%


\def\r{\mathbb{R}}
\def\n{\mathbb{N}}
\def\q{\mathbb{Q}}
\def\c{\mathbb{C}}
\def\z{\mathbb{Z}}

\def\sen{\mathop{\mbox{\normalfont sen}}\nolimits}
\def\intt{\mathop{\mbox{\normalfont int}}\nolimits}
\def\diag{\mathop{\mbox{\normalfont diag}}\nolimits}
\def\arcsen{\mathop{\mbox{\normalfont arcsen}}\nolimits}
\def\ln{\mathop{\mbox{\normalfont ln}}\nolimits}
\def\tr{\mathop{\mbox{\normalfont tr}}\nolimits}

%%%%%%%%%%%%%%%%%%%%%%%%%%%%%%%%%%%%%%

\begin{document}

%%%%%%%%%%%%%%%%%%%%%%%%%%%%%%%%%

\begin{minipage}{0.12\linewidth}
\includegraphics[width=20mm]{escudo.jpg}
\end{minipage}
\begin{minipage}{0.78\linewidth}
\large{\centerline{Universidad de Antioquia}}
\centerline{Facultad de Ciencias Exactas y Naturales}
\centerline{Instituto de Matem�ticas}
\centerline{Series de Tiempo II}
\end{minipage}

\vspace{5mm}

\rightline{Profesor: Duv{\'a}n Cata{\~n}o}
\rightline{25 de Septiembre de 2018}

\vspace{5mm}
\begin{minipage}{0.64\linewidth}
Nombre: \hrulefill
\end{minipage}
\begin{minipage}{0.3\linewidth}
\ Carn�: \hrulefill
\end{minipage}

\vspace{5mm}

%\fbox{
\begin{minipage}{0.95\linewidth}
\textbf{Nota:} \textit{El examen consta de 5 numerales para ser resueltos en un tiempo m�ximo de 2 horas. Los procedimientos empleados para llegar a cada respuesta deben ser justificados y quedar registrados en el examen.}
\end{minipage}
%%%%%%%%%%%%%%%%%%%%%%%%%%%%%%%%%%%%%%
\vspace*{5mm}


\begin{enumerate}


\item \textbf{(30$\%$)} Suponga que los residuos $\hat{a}_t$ del modelo $(1-B)x_t=(1+0,6B)a_t,$ ajustado de una serie de 80 observaciones, proporcionan las siguientes autocorrelaciones:
\bigskip
\begin{table}[htdp]
\begin{center}\begin{tabular}{|c|c|c|c|c|c|c|c|c|c|c|}
$h$ & 1 & 2 & 3 & 4 & 5 & 6 & 7 & 8 & 9 & 10  \\
\hline
$\rho_{\hat{a}}(h)$ & 0.39 & 0.20 & 0.09 & 0,04 & 0,09 & -0.08 & -0.05 & 0.06 & 0.07 & -0.02 
\end{tabular} 
\end{center}
\label{defaulttable}
\end{table} 
\begin{enumerate}
\item Analice la adecuaci{\'o}n del modelo ajustado y si existe alguna indicaci{\'o}n de falta de ajustamiento del modelo. Si esto ocurre, sugiera un modelo modificado.
\item Calcular la densidad espectral del modelo encontrado en el numeral anterior. Haga las suposiciones necesarias para garantizar su existencia.
\end{enumerate}

\bigskip

\item \textbf{(20$\%$)} Probar que $$\gamma(h)=\left\{\begin{array}{l}
1, \ \ h=0 \\
\rho,  \ \  h=\pm1 \\
0,  \ \  o.c\end{array}\right.$$
es una funci{\'o}n de autocovarianza si y s{\'o}lo si $|\rho|<1/2.$

\bigskip

\item \textbf{(30$\%$)} Sea $y_t=a_{t}+ca_{t-1}+ca_{t-2}+\ldots+ca_1$, para $t>0,$ donde $c\in\mathbb{R}$ y $a_t\sim RB(0, \sigma_a^2).$
\begin{enumerate}
 \item Calcular la media y autocovarianza de $y_t$. ?`Es estacionaria? 
 \item Demostrar que la serie $z_t=(1-B)y_t$ es estacionaria.
\item Calcular el espectro de $z_t$ y determinar la frecuencia donde alcanza el m{\'a}ximo.
\end{enumerate}
  
\bigskip

\item \textbf{(10$\%$)} Para el proceso $y_t=\sum_{j=-\infty}^{\infty}a_jx_{t-j}$, con $\sum_{j=-\infty}^{\infty}|a_j|<\infty.$ Demostrar que si $x_t$ tiene espectro $f_x(w)$, entonces el espectro de la serie filtrada de salida, $y_t$, $f_y(w)$, est{\'a} relacionada con el espectro de la serie de entrada $x_t$ mediante
$$f_y(w)=|A(w)|^2f_x(w)$$ donde la funci{\'o}n de respuesta frecuencia es dada $A(w)=\sum_{j=-\infty}^{\infty}a_je^{-2\pi iwj}.$

\bigskip

\item \textbf{(10$\%$)} Considere la serie $$x_t=\sin(2\pi Ut),$$ para $t=1, 2, \ldots,$ donde $U$ tiene distribuci{\'o}n uniforme sobre el intervalo $(0,1).$ Demostrar que $x_t$ no es estrictamente estacionaria. 



\end{enumerate}

\vspace{5mm}

\begin{minipage}{0.96\linewidth}
\hrulefill
\end{minipage}

\vspace{5mm}

\centerline{\textbf{Soluci�n}}

\end{document}
