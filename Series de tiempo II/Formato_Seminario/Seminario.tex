\documentclass[legalpaper,11pt]{article}

\pagestyle{empty}

\usepackage[ansinew]{inputenc}
\usepackage[spanish]{babel}
\usepackage[mathcal]{euscript}
\usepackage{amsmath,amsfonts,amssymb,theorem,latexsym,mathrsfs, %hyperref,
            epsfig, multicol,anysize,graphicx,enumitem,mdwlist}
\usepackage{graphicx}  
\usepackage{ragged2e}  
\usepackage{float}        

%%%%%%%%%%%%%%%%%%%%%%%%%%%%%%%%%

\marginsize{2cm}{2cm}{2cm}{2cm}

%%%%%%%%%%%%%%%%%%%%%%%%%%%%%%%%%%%%%%


\def\r{\mathbb{R}}
\def\n{\mathbb{N}}
\def\q{\mathbb{Q}}
\def\c{\mathbb{C}}
\def\z{\mathbb{Z}}

\def\sen{\mathop{\mbox{\normalfont sen}}\nolimits}
\def\intt{\mathop{\mbox{\normalfont int}}\nolimits}
\def\diag{\mathop{\mbox{\normalfont diag}}\nolimits}
\def\arcsen{\mathop{\mbox{\normalfont arcsen}}\nolimits}
\def\ln{\mathop{\mbox{\normalfont ln}}\nolimits}
\def\tr{\mathop{\mbox{\normalfont tr}}\nolimits}

%%%%%%%%%%%%%%%%%%%%%%%%%%%%%%%%%%%%%%

\begin{document}

%%%%%%%%%%%%%%%%%%%%%%%%%%%%%%%%%

\begin{minipage}{0.12\linewidth}
\includegraphics[width=25mm]{escudo.jpg}
\end{minipage}
\begin{minipage}{0.78\linewidth}
\large{\centerline{Universidad de Antioquia}}
\centerline{Facultad de Ciencias Exactas y Naturales}
\centerline{Instituto de Matem�ticas}
%\centerline{Formato para Seminario de Series de Tiempo II}
\end{minipage}

\begin{enumerate}

\vspace{0.5cm}
\begin{center}
\textbf{FORMATO PARA SEMINARIO DE SERIES DE TIEMPO I}
\end{center}
\vspace{1cm}

\item \textbf{T{\'I}TULO}\\
Resume y describe en t{\'e}rminos claros y precisos el contenido esencial del texto.
\medskip

\item \textbf{RESUMEN}\\
Presenta una s{\'i}ntesis del contenido, resultados y conclusiones del trabajo realizado.
\medskip

\item \textbf{PALABRAS CLAVES}\\
Conceptos fundamentales utilizados en el trabajo.
\medskip

\item \textbf{INTRODUCCI{\'O}N}\\
Presenta el trabajo realizado, describe en general la metodolog{\'i}a utilizada y plantea los objetivos del trabajo.
\medskip

\item \textbf{MARCO TE{\'O}RICO}\\
Presenta los principales conceptos y teor{\'i}as que sustentan el trabajo, a la luz de referencias bibliogr{\'a}ficas. 
\medskip

 \item \textbf{OBJETIVOS}\\
El objetivo general se{\~n}ala la meta principal que se desea alcanzar con el trabajo, mientras que los objetivos espec{\'i}ficos se refieren a los pasos sucesivos a seguir  para alcanzar el objetivo general.
\medskip

\item \textbf{METODOLOG{\'I}A}\\
Define el tipo de dise{\~n}o metodol{\'o}gico elegido para el trabajo;  describe las herramientas empleadas en los datos  utilizados.
\medskip

\item \textbf{APLICACI{\'O}N}\\
Describe los datos y presenta los resultados obtenidos al aplicar la metodolog{\'i}a estudiada.  Debe contener tanto texto explicativo como gr{\'a}ficos ilustrativos.
\medskip

\item \textbf{CONCLUSIONES}\\
Confronta la teor{\'i}a utilizada en el trabajo, con los resultados de la aplicaci{\'o}n. Presenta ventajas y desventajas de la metodolog{\'i}a en consideraci{\'o}n.

\medskip

\item \textbf{REFERENCIAS BIBLIOGR{\'A}FICAS}\\
Se presentan en orden alfab{\'e}tico todas las fuentes, tanto impresas como de Internet.


\end{enumerate}


\end{document}
